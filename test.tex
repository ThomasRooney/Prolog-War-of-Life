\documentclass[a4wide, 11pt]{article}
\usepackage{a4, fullpage}
\newcommand{\tab}{\hspace*{2em}}
\usepackage[margin=2cm]{geometry}
\newcommand{\tx}{\texttt}

\begin{document}

\title{AI Coursework}
\author{Thomas Rooney}
\date{\today}
\maketitle

\begin{verbatim}
        Introduction to Artificial Intelligence Coursework[1]

       (Submission by groups of two students allowed/welcome)
                  Due date: Thursday 7  March 2013
        Electronic submission (code + this worksheet) on CATE
\end{verbatim}

\begin{enumerate}
\item  Introduction
\end{enumerate}

We will be looking at a two player board game called ``war of life''
which will be played on an 8x8 board. Player 1 will start with a random
configuration of 12 blue pieces and player 2 will start with a similar
random configuration of 12 red pieces. An example initial configuration
might be (where b stands for ``blue piece'' and r stands for ``red
piece''):

\textbar{} \textbar{} The game terminates as follows: • If at some stage
no (red or blue) pieces at all are left on the board, then the game is
drawn. • If, when it is his/her turn, a player cannot move anywhere,
then the game is declared a stalemate and is drawn. • If one player has
no pieces left on the board, then that player loses and the other player
wins. • If the game lasts for 250 moves without a winner, then it is
declared an exhausted draw.

Part 1 -- Getting to know the Game

Question 1

\textbar{} \textbar{}

Board states are represented in the program as pairs of lists, where the
first list contains the co-ordinates of all the alive blue pieces and
the second list contains the co-ordinates of all the alive red pieces.
For example, this is a simple board state with two alive blues and one
alive red:

{[}{[}{[}3,4{]},{[}5,7{]}{]},{[}{[}8,8{]}{]}{]}

Question 2

In the box below, write down the Prolog representation for the initial
board state given in question 1.

\textbar{} \textbar{}

Use this representation, and the predicate

next\_generation(+BoardState, -NextGenerationBoardState)

to check whether your answer for question 1 was correct. If it is, then
run a further two generations, and put the resulting board states in the
tables below:

\textbar{} \textbar{}1 \textbar{} \textbar{}Number of wins for player 1
(blue) \textbar{} \textbar{} \textbar{}Number of wins for player 2 (red)
\textbar{} \textbar{} \textbar{}Longest (non-exhaustive) game \textbar{}
\textbar{} \textbar{}Shortest game \textbar{} \textbar{}
\textbar{}Average game length (including \textbar{} \textbar{}
\textbar{}exhaustives) \textbar{} \textbar{} \textbar{}Average game time
\textbar{} \textbar{}

Question 4

Does it look like there is any advantage to playing first if both
players choose moves randomly? Answer in the space below.

Part 2 -- Implementing Strategies

In this part, we will be implementing search strategies in order to
improve a war of life playing agent's performance. They already have one
strategy: random, which chooses any piece randomly and moves it
randomly. The question is: can we implement any strategy which
out-performs this one?

We will look at four strategies:

Bloodlust: This strategy chooses the next move for a player to be the
one which (after Conway's crank) produces the board state with the
fewest number of opponent's pieces on the board (ignoring the player's
own pieces).

Self Preservation: This strategy chooses the next move for a player to
be the one which (after Conway's crank) produces the board state with
the largest number of that player's pieces on the board (ignoring the
opponent's pieces).

Land Grab: This strategy chooses the next move for a player to be the
one which (after Conway's crank) produces the board state which
maximises this function: Number of Player's pieces -- Number of
Opponent's pieces.

Minimax: This strategy looks two-ply ahead using the heuristic measure
described in the Land Grab strategy. It should follow the minimax
principle and take into account the opponent's move after the one being
chosen for the current player.

In your file my\_wol.pl, write five predicates:

bloodlust(+PlayerColour, +CurrentBoardState, -NewBoardState, -Move).
self\_preservation(+PlayerColour, +CurrentBoardState, -NewBoardState,
-Move). land\_grab(+PlayerColour, +CurrentBoardState, -NewBoardState,
-Move). minimax(+PlayerColour, +CurrentBoardState, -NewBoardState,
-Move).

These predicates will implement the four strategies described above by
choosing the next move for a player. They will all do the same thing:
choose the next move for the player. PlayerColour will either be the
constant b for blue or r for red. The CurrentBoardState will be the
state of the board upon which the move choice is going to be made. The
predicate will produce a NewBoardState, in the usual representation
(pair of lists) which will represent the board state after the move, but
before Conway's crank is turned. The predicate will also return the Move
that changed the current to the new board state. This will be
represented as a list {[}r1,c1,r2,c2{]} where the move chosen is to move
the piece at co-ordinate (r1, c1) to the empty cell at co-ordinate (r2,
c2).

It should be possible to run these strategies in the same way as the
random strategy, using the constants bloodlust, self\_preservation,
land\_grab. For example, if you load war\_of\_life.pl and my\_wol.pl
into Prolog, then type the query:

play(verbose, bloodlust, land\_grab, NumberOfMoves, WinningPlayer).

this will play a game in which player 1 uses the bloodlust strategy and
player 2 uses the land\_grab strategy. Check that this is working OK by
running a few games with different strategy pairings.

Part 3 -- A Tournament

Question 6

We want to know which, if any, strategy is the best for playing this
game, and we'll do this by using a tournament. Play as many games as
time will allow for each pairing of strategies, and fill in the table
over the page.

Question 7

If both players have the same strategy, for which strategies does it
appear that playing first is an advantage/disadvantage? Answer in the
space below:

Question 8

Imagine playing football in a gale force wind and where the pitch is
extremely muddy. Here, it is hardly worth the players kicking the ball,
because the environment plays too big a factor in the game. What
evidence do you have against the claim that the environment plays a
bigger part than the movement of pieces in the war of life game? The
environment here is Conway's Crank, which is beyond the control of the
players. Answer in the space below:

Submitting Your Work

Electronic submission: submit a zipped directory WOL.zip containing 1) a
my\_wol.pl file and 2) a file worksheet.pdf (your completed version of
this worksheet).

Note: my\_wol.pl should run under Sicstus on the lab machines -- do not
use SWI or other Prologs!

\textbar{}P1 strategy \textbar{}P2 strategy \textbar{}Games \textbar{}P1
\textbar{}P2 \textbar{}Draws \textbar{}Av. \textbar{}Av. Game\textbar{}
\textbar{} \textbar{} \textbar{}Played\textbar{}wins \textbar{}wins
\textbar{} \textbar{}Game \textbar{}Time \textbar{} \textbar{}
\textbar{} \textbar{} \textbar{} \textbar{} \textbar{} \textbar{}length
\textbar{} \textbar{} \textbar{}Random \textbar{}Bloodlust \textbar{}
\textbar{} \textbar{} \textbar{} \textbar{} \textbar{} \textbar{}
\textbar{}Random \textbar{}Self Preservation\textbar{} \textbar{}
\textbar{} \textbar{} \textbar{} \textbar{} \textbar{} \textbar{}Random
\textbar{}Land Grab \textbar{} \textbar{} \textbar{} \textbar{}
\textbar{} \textbar{} \textbar{} \textbar{}Random \textbar{}Minimax
\textbar{} \textbar{} \textbar{} \textbar{} \textbar{} \textbar{}
\textbar{} \textbar{}Bloodlust \textbar{}Random \textbar{} \textbar{}
\textbar{} \textbar{} \textbar{} \textbar{} \textbar{}
\textbar{}Bloodlust \textbar{}Bloodlust \textbar{} \textbar{} \textbar{}
\textbar{} \textbar{} \textbar{} \textbar{} \textbar{}Bloodlust
\textbar{}Self Preservation\textbar{} \textbar{} \textbar{} \textbar{}
\textbar{} \textbar{} \textbar{} \textbar{}Bloodlust \textbar{}Land Grab
\textbar{} \textbar{} \textbar{} \textbar{} \textbar{} \textbar{}
\textbar{} \textbar{}Bloodlust \textbar{}Minimax \textbar{} \textbar{}
\textbar{} \textbar{} \textbar{} \textbar{} \textbar{} \textbar{}Self
Preservation\textbar{}Random \textbar{} \textbar{} \textbar{} \textbar{}
\textbar{} \textbar{} \textbar{} \textbar{}Self
Preservation\textbar{}Bloodlust \textbar{} \textbar{} \textbar{}
\textbar{} \textbar{} \textbar{} \textbar{} \textbar{}Self
Preservation\textbar{}Self Preservation\textbar{} \textbar{} \textbar{}
\textbar{} \textbar{} \textbar{} \textbar{} \textbar{}Self
Preservation\textbar{}Land Grab \textbar{} \textbar{} \textbar{}
\textbar{} \textbar{} \textbar{} \textbar{} \textbar{}Self
Preservation\textbar{}Minimax \textbar{} \textbar{} \textbar{}
\textbar{} \textbar{} \textbar{} \textbar{} \textbar{}Land Grab
\textbar{}Random \textbar{} \textbar{} \textbar{} \textbar{} \textbar{}
\textbar{} \textbar{} \textbar{}Land Grab \textbar{}Bloodlust \textbar{}
\textbar{} \textbar{} \textbar{} \textbar{} \textbar{} \textbar{}
\textbar{}Land Grab \textbar{}Self Preservation\textbar{} \textbar{}
\textbar{} \textbar{} \textbar{} \textbar{} \textbar{} \textbar{}Land
Grab \textbar{}Land Grab \textbar{} \textbar{} \textbar{} \textbar{}
\textbar{} \textbar{} \textbar{} \textbar{}Land Grab \textbar{}Minimax
\textbar{} \textbar{} \textbar{} \textbar{} \textbar{} \textbar{}
\textbar{} \textbar{}Minimax \textbar{}Random \textbar{} \textbar{}
\textbar{} \textbar{} \textbar{} \textbar{} \textbar{} \textbar{}Minimax
\textbar{}Bloodlust \textbar{} \textbar{} \textbar{} \textbar{}
\textbar{} \textbar{} \textbar{} \textbar{}Minimax \textbar{}Self
Preservation\textbar{} \textbar{} \textbar{} \textbar{} \textbar{}
\textbar{} \textbar{} \textbar{}Minimax \textbar{}Land Grab \textbar{}
\textbar{} \textbar{} \textbar{} \textbar{} \textbar{} \textbar{}
\textbar{}Minimax \textbar{}Minimax \textbar{} \textbar{} \textbar{}
\textbar{} \textbar{} \textbar{} \textbar{}

\begin{center}\rule{3in}{0.4pt}\end{center}

\end{document}
